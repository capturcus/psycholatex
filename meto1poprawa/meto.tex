\documentclass[12pt]{article}
\usepackage[margin=2.5cm]{geometry}
\usepackage{polski}
\usepackage[utf8]{inputenc}
\usepackage{qtimes}
\usepackage{setspace}
\usepackage{amssymb}
\doublespacing

\title{Paracetamol zmniejsza odporność na ból}
\author{Marcin Parafiniuk}
\date{\today}

\newcommand{\bb}[1]{$\mathbb{#1}$}
\newcommand{\apaartykul}[7]{#1 (#2). #3. \textit{#4}, \textit{#5} (#6), s. #7}

\begin{document}

\maketitle

Artykuł źródłowy: 
\apaartykul{Mischkowski, D., Crocker, J., Way, M. B.}{2016}{From painkiller to empathy killer: acetaminophen (paracetamol) reduces empathy for pain}{Social Cognitive and Affective Neuroscience}{11}{9}{1345–1353}

\section{Proces od pomysłu do hipotezy}

\bb{U}miejętność empatyzowania z osobą, której dzieje się fizyczna krzywda, jest jednym z najbardziej
podstawowych mechanizmów społecznych. Nic dziwnego więc, że mechanizm ten zainteresował grupę
amerykańskich psychologów. Zauważyli oni, że zarówno odbieranie bólu, jak i empatyzowanie z
czyimś bólem aktywuje te same obszary w mózgu. Dodatkowo, branie leków przeciwbólowych
wpływa na zmniejszenie aktywności tego ośrodka. Łącząc te dwa fakty, dostajemy pytanie badawcze –
czy przyjmowanie środków chemicznych takich jak paracetamol zmniejszy nie tylko percepcję
własnego bólu, ale także i czyjegoś?

\section{Hipotezy badawcze}

\section{Zmienne}

\begin{tabular}{|r|l|}
  \hline
  \textbf{Zmienna (rodzaj zmiennej)} & \textbf{Wskaźnik zmiennej (skala pomiaru)} \\
  \hline
  7C0 & hexadecimal \\
  \hline
  3700 & octal \\
  \hline
  11111000000 & binary \\
  \hline
  1984 & decimal \\
  \hline
\end{tabular}

\end{document}