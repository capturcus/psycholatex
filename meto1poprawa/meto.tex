\documentclass[12pt]{article}
\usepackage[margin=2.5cm]{geometry}
\usepackage{polski}
\usepackage[utf8]{inputenc}
\usepackage{qtimes}
\usepackage{setspace}
\usepackage{amssymb}
\doublespacing

\title{Paracetamol zmniejsza odporność na ból}
\author{Marcin Parafiniuk}
\date{\today}

\newcommand{\bb}[1]{$\mathbb{#1}$}
\newcommand{\apaartykul}[7]{#1 (#2). #3. \textit{#4}, \textit{#5} (#6), s. #7}

\begin{document}

\maketitle

Artykuł źródłowy: 
\apaartykul{Mischkowski, D., Crocker, J., Way, M. B.}{2016}{From painkiller to empathy killer: acetaminophen (paracetamol) reduces empathy for pain}{Social Cognitive and Affective Neuroscience}{11}{9}{1345–1353}

\section{Proces od pomysłu do hipotezy}

\bb{U}miejętność empatyzowania z osobą, której dzieje się fizyczna krzywda, jest jednym z najbardziej
podstawowych mechanizmów społecznych. Nic dziwnego więc, że mechanizm ten zainteresował grupę
amerykańskich psychologów. Zauważyli oni, że zarówno odbieranie bólu, jak i empatyzowanie z
czyimś bólem aktywuje te same obszary w mózgu. Dodatkowo, branie leków przeciwbólowych
wpływa na zmniejszenie aktywności tego ośrodka. Łącząc te dwa fakty, dostajemy pytanie badawcze –
czy przyjmowanie środków chemicznych takich jak paracetamol zmniejszy nie tylko percepcję
własnego bólu, ale także i czyjegoś?

\section{Hipotezy badawcze}

Paracetamol wpływa na zmniejszenie odczuwania dyskomfortu w związku z empatycznym postrzeganiem czyjegoś bólu

\section{Zmienne}


\begin{tabular}{|p{9cm}|l|l|}
  \hline
  \textbf{Zmienna} & \textbf{rodzaj zmiennej} & \textbf{Wskaźnik (skala pomiaru)} \\
  \hline
  \multicolumn{3}{|c|}{afektywność wg charakterystyki PANAS } \\
  \hline
  pozytywna & zależna & 1-5 (porządkowa) \\
  negatywna & zależna & 1-5 (porządkowa) \\
  \hline
  \multicolumn{3}{|c|}{czytanie historii o bólu fizycznym} \\
  \hline
  empatia poznawcza & zależna & 1-5 (porządkowa) \\
  empatia emocjonalna & zależna & 1-5 (porządkowa) \\
  \hline
  \multicolumn{3}{|c|}{czytanie historii o bólu społecznym} \\
  \hline
  empatia poznawcza & zależna & 1-5 (porządkowa) \\
  empatia emocjonalna & zależna & 1-5 (porządkowa) \\
  \hline
  \hline
  przyjmowanie paracetamolu / placebo & główna & nominalna \\
  \hline
  rodzaj czytanej historii - ból fizyczny / psychiczny & moderator & nominalna \\
  \hline
  losowy dyskomfort ("noise unpleasantness") & mediator & 0.0 -- 1.0 (interwałowa) \\
  \hline
  możliwość odróżnienia paracetamolu od placebo & zakłócająca & nominalna \\
  przyjmowanie dużych dawek paracetamolu w życiu codziennym & zakłócająca & nominalna \\
  \hline
  płeć badanych & wyeliminowana & nominalna \\
  pochodzenie & wyeliminowana & nominalna \\
  wiek & wyeliminowana & interwałowa \\
  kolejność scenariuszy & wyeliminowana & nominalna \\
  płeć bohaterów scenariuszy & wyeliminowana & nominalna \\
  \hline
\end{tabular}

\end{document}