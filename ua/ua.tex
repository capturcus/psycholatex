\documentclass[12pt]{article}
\usepackage[margin=2.5cm]{geometry}
\usepackage{polski}
\usepackage[utf8]{inputenc}
\usepackage{qtimes}
\usepackage{setspace}
\usepackage{amssymb}
\doublespacing

\title{O komunikacji szczurów}
\author{Marcin Parafiniuk}
\date{\today}

\newcommand{\apaartykul}[7]{#1 (#2). #3. \textit{#4}, \textit{#5} (#6), s. #7}

\begin{document}

\maketitle

\tableofcontents

\pagebreak

\section{Wstęp}

Szczury to zwierzęta, które towarzyszą człowiekowi od dawna. Szczur, tak jak wróbel czy gołąb, jest jednym z gatunków, które świetnie czują się w środowisku zurbanizowanym. W osiemnastowiecznej Europie szczur był jednym z najpopularniejszych szkodników gnieżdżących się w miastach. Stanowiło to na tyle duży problem, że powstał cały poddział przemysłu, trudniący się zwalczaniem tychże szkodników. Szczurołapowie, oprócz zarabiania na zwalczaniu gryzoni, obławiali się na sprzedawaniu szczurzego mięsa, lub na tzw. szczurzych igrzyskach.

Szczurze igrzyska polegały na tym, że złapane gryzonie wpuszczało się na arenę razem z jednym terrierem. Zebrani naokoło widzowie mogli zakładać się o to, jak szybko pies zabije wszystkie szkodniki. Ten całkiem popularny sport, zanim został zakazany, przyciągał tłumy. Psy które zabijały szczury najszybciej, zostawały lokalnymi celebrytami, ich właścicielowie szybko się wzbogacali. Niedługo okazało się, że szczurołapowie nie nadążają z łapaniem gryzoni na potrzeby igrzysk i organizatorzy takich wydarzeń byli zmuszeni zainwestować w hodowlę szczurów.

Hodowle radykalnie różniły się warunkami od tego do czego gryzonie były przyzwyczajone w naturalnym środowisku, miejskim, czy leśnym. Dobór naturalny został zakłócony, często miał miejsce chów wsobny. Wskutek tego wśród szczurów zaczęły pojawiać się mutacje. W ten sposób pojawiły się albinosy i osobniki o tzw. umaszczeniu kapturowym. Ludzie zainteresowali się różnicami pomiędzy zwykłymi szczurami, a dziwnymi białymi wyjątkami pochodzącymi z hodowli. W 1828 pierwszy raz szczur albinos wziął udział w eksperymencie dotyczącym poszczenia. Podczas następnych 30 lat szczury były używane jako obiekty doświadczalne w kolejnych eksperymentach i w końcu szczur laboratoryjny został udomowiony ze względów naukowych.

Szczury są droższe w utrzymaniu niż myszy, dlatego głównym obiektem badań laboratoryjnych wciąż pozostają mniejsze gryzonie. Jednak w niektórych polach szczury zdecydowanie przodują w statystykach nad swoim mniejszym krewniakiem. Takim polem jest między innymi psychologia, jako że szczury charakteryzują się silniejszym instynktem stadnym niż myszy, są też bardziej inteligentne.

\pagebreak

W środowisku naturalnym stado szczurów może liczyć nawet 200 osobników. Jasne jest, że tak duże kolonie nie mogą istnieć bez jakiegoś sposobu, na który osobniki mogą się ze sobą porozumiewać. Zafascynował mnie ten temat. W jaki sposób owe złożone, inteligentne zwierzęta utrzymują jedność stada? Jak skomplikowane są komunikaty, które sobie nawzajem przekazują? Czy ze szczurem da się w jakiś sposób "porozmawiać" - a jeśli tak, to o czym? Odpowiedzi na te pytania, chociażby niepełne, postarałem się zawrzeć w tej pracy.

\section{Biochemiczne podłoże zachowań społecznych szczurów}

\subsection{Opioidy}

Pomysł na to, że endogenne opioidy regulują zachowania społeczne ma swoje źródło w tym, że występują istotne podobieństwa pomiędzy uzależnieniem od opiatów, a potrzebą kontaktu osobnika z innymi szczurami. W obu przypadkach zachodzi faza początkowego przywiązania i faza budowania tolerancji, podobne są też objawy odstawienne (Panksepp, J., Herman, B.H., Conner, R., Bishop, P., Scott, J.P., 1978; Panksepp, J., Herman, B.H., Vilberg, T., Bishop, P., DeEskinazi, F.G., 1980). Wykazano też, że podawanie morfiny zmniejsza częstotliwość ultradźwiękowych wokalizacji wywołanych przez izolację dwutygodniowych szczurów od matki -- młode w tym wieku wciąż powinny być karmione piersią i ich relacja z matką jest bardzo bliska (Carden, S.E., Barr, G.A., Hofer, M.A., 1991).
Jednak u młodych szczurów dawki, które redukują ultradźwiękowe wokalizacje, wpływają również na ogólne zmniejszenie aktywności. Młode gryzonie różnią się zarówno od starszych przedstawicieli tego samego gatunku jak i od innych gatunków zdecydowanie większą wrażliwością na opioidy podczas dwóch pierwszych tygodni życia, prawdopodobnie ze względu na większą przenikalność bariery krew-mózg (Banks, W.A., Kastin, A.J., 1985).

Badania dotyczące antagonistów, które hamują działanie endogennie wydzielanych opioidów również wskazują na specyfikę tych substancji w regulowaniu zachowań społecznych. Nalokson i naltrekson blokują pozytywny wpływ opioidów na ilość wokalizacji. Podane same (bez równoważącej dawki opioidów) często zwiększają efekt społecznej izolacji (Panksepp, J. i in., 1980).

Niektóre badania wskazują na to, że niektóre bodźce społeczne, na przykład bycie karmionym piersią, również wiąże się z endogennymi opioidami. U nowonarodzonych szczurów występuje zmniejszenie wrażliwości na dotykowe bodźce zewnętrzne podczas karmienia piersią. Ten efekt jest blokowany przez podanie małemu szczurowi opioidowego antagonisty, wskazując na to że reakcja na bycie karmionym piersią obejmuje wydzielanie endogennych opioidów (Smotherman, W.P., Robinson, S.R., 1992).

Podczas okresu dojrzewania szczury, tak jak praktycznie wszystkie gatunki ssaków bawią się w gonitwy i przepychanki. Wykazano, że zasadniczym celem takiego zachowania są interakcje somatosensoryczne (Siviy, S.M. and Panksepp, J., 1987). Badania przy pomocy subtraktywnej autoradiografii wykazały że szczurze zabawy skutkują wydzielaniem endogennych opioidów, oraz że podanie szczurom naloksonu wpływa na zmniejszenie poziomu tych zabaw (Panksepp, J., Bishop, P., 1987; J. Panksepp, S.M. Siviy, L.A. Normansell, 1985).

\subsection{Oksytocyna}

Wykazano, że podanie oksytocyny zmniejsza częstotliwość ultradźwiękowych wokalizacji u młodych szczurów (Insel, T.R. and Winslow, J.T., 1991). To odkrycie sugeruje, że oksytocyna może mieć zasadnicze znaczenie w "spokoju", który młode okazują podczas interakcji z matką i innymi szczurami. Co więcej, inne badanie wykazało że podanie antagonisty oksytocyny blokuje efekt kojarzenia zapachu matki z czymś przyjemnym u młodych szczurów (Nelson, E. and Panksepp, J., 1996).

W innym badaniu młode we wczesnej fazie rozwoju zostawały kilkukrotnie poddawane izolacji od matki. Okazało się, że ich receptory oksytocyny w hipokampie zostały zredukowane (Noonan, L.R., Caldwell, J.D., Li, L., Walker, C.H., Pedersen, C.A. and Mason, G.A., 1994). Chociaż dokładne znaczenie tego odkrycie jest niejasne, na pewno wskazuje ono na dużą rolę oksytocyny w interakcjach społecznych z matką u młodych szczurów.

\pagebreak

Prawdopodobnym jest, że tak jak opioidy, oksytocynowe neurony są aktywowane przez jakikolwiek społeczny kontakt fizyczny. Niedawne badanie wykazało, że poziom oksytocyny w krwi i płynie mózgowo-rdzeniowym wzrasta podczas głaskania lub delikatnego ogrzewania szczura (Uvnas-Moberg, K., Bruzelius, G., Alster, P. and Lundeberg, T., 1993).

\subsection{Wazopresyna}

Chociaż nie tak popularna w literaturze jak oksytocyna, wazopresyna również może pełnić istotną rolę w komunikacji szczurów. Hormon ten, podobnie jak poprzednio wymienione substancje, zmniejsza częstotliwość ultradźwiękowych wokalizacji u młodych oddzielonych od matki (Winslow, J.T., Insel, T.R., 1993). Co więcej, u szczepu Brattleboro, charakteryzującego się naturalnym deficytem wazopresyny, wykazano zmniejszoną reakcję u młodych na zapach matki (Nelson, E., Bird, L., Deak, T., Vaningan, M. Panksepp, J., 1995).

\ifx true false

Although not as extensive as the oxytocin literature, a
similar affiliative role has also been suggested for the
closely related nonapeptide vasopressin. Vasopressin
reduces USV frequency in isolated young rat pups (215),
and an attenuated maternally associated odor preference has
been found in the vasopressin-deficient Brattleboro strain of
rat ((125), and see below). These findings suggest that
vasopressin, like oxytocin may be involved in emergent
affiliative processes in young rats.

\fi

%Nelson, E., Bird, L., Deak, T., Vaningan, M. and Panksepp, J., Social behavior in the young, vasopressin deficient Brattleboro rat. Soc. Neurosci. Abstr., 1995, 20, 366.

%Winslow, J.T. and Insel, T.R., Effects of central vasopressin adminis-tration to infant rats. Eur. J. Pharmacol., 1993, 233, 101–107.

%Uvnas-Moberg, K., Bruzelius, G., Alster, P. and Lundeberg, T., The antinociceptive effect of non-noxious sensory stimulation is mediated partly through oxytocinergic mechanisms. Acta Physiol. Scand., 1993, 149, 199–204.

%Noonan, L.R., Caldwell, J.D., Li, L., Walker, C.H., Pedersen, C.A. and Mason, G.A., Neonatal stress transiently alters the development of hippocampal oxytocin receptors. Dev. Brain Res., 1994, 80, 115–120.

%Nelson, E. and Panksepp, J., Oxytocin mediates acquisition of maternally associated odor preference in preweanling rat pups. Behav. Neurosci., 1996, 110, 1–10.

% Insel, T.R. and Winslow, J.T., Central administration of oxytocin modulates the infant rat’s response to social isolation. Eur. J. Pharmacol., 1991, 203, 149–152.

%J. Panksepp, S.M. Siviy, L.A. Normansell, Brain opioids and social emotion, in: M. Reite, T. Fields (Eds.), The Psychobiology of Attachment and Separation, Academic, New York, 1985, pp. 3–49.

%Panksepp, J., Bishop, P., An autoradiographic map of (3H) diprenorphine binding in rat brain: Effects of social interaction. Brain Res. Bull., 1981, 7, 405–410.

%Siviy, S.M., Panksepp, J., Sensory modulation of juvenile play in rats. Dev. Psychobiol., 1987, 20, 39–55.

%Smotherman, W.P. and Robinson, S.R., Kappa opioid mediation of fetal responses to milk. Behav. Neurosci., 1992, 106, 396–407.

%Banks, W.A. and Kastin, A.J., Permeability of the blood-brain barrier to neuropeptides: The case for penetration. Psychoneuro-endocrinology, 1985, 10, 385–399.

%Carden, S.E., Barr, G.A. and Hofer, M.A., Differential effects of specific opioid receptor agonists on rat up isolation calls. Behav. Brain Res., 1991, 62, 17–22.

%Panksepp, J., Herman, B.H., Conner, R., Bishop, P. and Scott, J.P., The biology of social attachments: opiates alleviate separation distress. Biol. Psychiat., 1978, 13, 607–613.

%Panksepp, J., Herman, B.H., Vilberg, T., Bishop, P. and DeEskinazi, F.G., Endogenous opioids and social behavior. Neurosci. Biobehav. Rev., 1980, 4, 473–487.

\end{document}