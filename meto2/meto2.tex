\documentclass{psychol}

\title{Recenzja artykułu naukowego: Sinacka-Kubik, E. (2011). Wpływ wstrętu na uprzedzenia etniczne kobiet i mężczyzn. Psychologia Społeczna, 6(1), 24-33.}
\author{Marcin Parafiniuk}
\newcommand{\album}{347235} %jak nie ma numeru albumu to trzeba zakomentować
\newcommand{\kierownik}{dr. Joanny Kowalskiej}
\date{\today}
\newcommand{\katedra}{Katedra Psychologii Zdrowia i Rehabilitacji}
\newcommand{\przedmiot}{Metodologię Badań Psychologicznych}

\begin{document}

\maketitle
\doublespacing

\pagebreak

\Section{Streszczenie}

Badanie dotyczyło wpływu wywołanych uprzednio emocji na uprzedzenia względem obcych grup etnicznych.

\subsection{Postawione hipotezy badawcze:}
\begin{itemize}
    \item Dystans psychologiczny wobec osób z wybranych grup etnicznych jest największy pod wpływem manipulacji wstrętem w porównaniu z innymi emocjami negatywnymi jak lęk czy smutek oraz w porównaniu z manipulacją bodźcami neutralnymi.
    \item Wzbudzenie wstrętu u osób badanych powoduje większy dystans psychologiczny wobec osób z grupy obcej w porównaniu z grupą, do której należy badany.
    \item Wzbudzenie wstrętu powoduje większy dystans psychologiczny wobec Afroamerykanów, Arabów i Żydów niż wobec Meksykanów i Azjatów.
    \item Kobiety w porównaniu z mężczyznami reagują większym dystansem psychologicznym wobec grupy obcej.
\end{itemize}

\costam[23456][yyyyyy]

Badanie było anonimowe i dobrowolne. W  eksperymencie wzięło udział 96 studentów (48 kobiet i 48 mężczyzn) Uniwersytetu Gdańskiego w wieku 21–34 lata. Badanie było korelacyjne. Była wyznaczona grupa kontrolna. W eksperymencie posłużono się specjalnie skonstruowanym programem komputerowym, modyfikacją metody symulowanego dążenia – unikania. Na ekranie komputera o rozdzielczości 600 x 800 pikseli w punkcie początkowym o współrzędnych (400, 480)
umieszczony był „ludzik”, symbolizujący osobę badaną.
Poniżej znajdowało się zdjęcie z obiektem o współrzędnych środka zdjęcia (400, 200), do którego miał się ustosunkować badany przez przesunięcie „ludzika” w górę
(odsunięcie się od obiektu) lub w dół (przysunięcie się
do obiektu). Odległość „ludzika” od zdjęcia było interpretowane jako stopień dystansu psychologicznego. W badaniu brało udział 12 zdjęć ludzi z następujących grup etnicznych: Azjaci, Afroamerykanie, Arabowie,
Żydzi, Meksykanie i Europejczycy, z każdej po dwa zdjęcia. Do tego, żeby odpowiednio wywołać emocje przed oglądaniem zdjęć grup etnicznych, przed ekspozycją na fotografie osób pokazywano badanym zdjęcia obiektów wywołujących smutek, wstręt lub lęk.

\subsection{Badane zmienne:}

\begin{itemize}
    \item Niezależne:
    \begin{itemize}
        \item płeć
        \item rodzaj emocji wywoływanej w badanym przed ekspozycją na fotografie osób: smutek, lęk, wstręt, neutralna
        \item rodzaj grupy etnicznej do której należy osoba na fotografii: Azjaci, Afroamerykanie, Arabowie,
        Żydzi, Meksykanie i Europejczycy
        \item to, czy grupa etniczna do której należy osoba na fotografii należy do grup powszechnie w Polsce postrzeganych pejoratywnie
    \end{itemize}
    \item {Zależne:}
    \begin{itemize}
        \item dystans psychologiczny \costam[1][yyyyyy] (mierzony jako odległość ludzika na ekranie od środka zdjęcia w pikselach). \costam[3][yyyyyy]
    \end{itemize}
    \item {Kontrolowane:}
    \begin{itemize}
        \item pochodzenie uczestników (wszyscy byli z Uniwersytetu Gdańskiego)
        \item wiek uczestników
    \end{itemize}
\end{itemize}

\subsection{Wyniki badania}

Wyniki analizy wariancji wskazują, że dystans psychologiczny wobec osób z wybranych grup etnicznych różni
się w zależności od wzbudzonej emocji. Reakcja badanych na wybrane grupy etniczne pod wpływem manipulacji bodźcami neutralnymi różni się istotnie od reakcji pod wpływem wstrętu i lęku, natomiast nie różni się od reakcji pod wpływem smutku. Dystans psychologiczny badanych pod wpływem wstrętu różni
się istotnie w porównaniu z dystansem psychologicznym zarejestrowanym po wzbudzeniu lęku, smutku oraz manipulacji bodźcami neutralnymi. Reakcja na wybrane grupy etniczne zarejestrowana pod wpływem lęku różni się istotnie od reakcji
zarejestrowanej pod wpływem smutku, wstrętu i bodźców neutralnych.

Analiza statystyczna przeprowadzona wykazała, że dystans psychologiczny zarejestrowany pod wpływem manipulacji
wstrętem wobec grupy obcej jest istotnie większy w porównaniu z dystansem do grupy \costam[123][zzzzzzz] \costam[321][zzzzzzz]
własnej. Wyniki wskazują też, że kobiety i mężczyźni różnią się istotnie w reakcjach na osoby z wybranych grup etnicznych pod wpływem wstrętu. Dystans psychologiczny zarejestrowany u kobiet jest znacznie większy w porównaniu z dystansem zarejestrowanym
u mężczyzn.

\Section{Ocena poprawności badania}

Badanych było niedużo, mniej niż 100. Zostali też wybrani ze stosunkowo specyficznej grupy społecznej, jaką są studenci Uniwersytetu Gdańskiego. Są to więc ludzie młodzi, mieszkający w miastach i nastawieni na zdobywanie wykształcenia wyższego. Pominięto tutaj przynajmniej dwie (nierozłączne) istotne grupy społeczne. Według badań, na terenach wiejskich mieszka prawie 40\% ludności w Polsce, a procent ten na przestrzeni lat wciąż rośnie. Myślę, że pomijanie tak dużego procentu ludności polskiej jest istotnym błędem, szczególnie że mentalność wiejska na pewno różni się drastycznie od mentalności miejskiej w kwestiach tolerancji. Drugą grupą społeczną, która została pominięta w badaniu są ludzie starsi. Pomijając ludzi starszych niż 34 lata, ignorujemy ponad 57 procent społeczeństwa! Dosyć ryzykowne wydaje mi się wyciąganie ogólnych wniosków na podstawie badań przeprowadzonych na grupie nie zawierającej przedstawicieli tak licznej grupy społecznej. Dodatkowo, tak samo jak w przypadku pochodzenia wiejskiego lub miejskiego, myślę że wiek wpływałby znacznie na badania dotyczące tolerancji.

Inną kwestią, na którą chciałbym zwrócić uwagę, jest procedura przeprowadzania badania. Znamy rozmiar ekranu, znamy współrzędne umieszczenia środka zdjęcia, ale nie zostało wspomniane czy zdjęcia są tej samej wielkości. Na pewno to, czy „ludzik” wszedłby na obszar zdjęcia wpłynęłoby na to, jak blisko byśmy byli skłonni zbliżyć się nim do zadanego punktu. Musimy też wziąć pod uwagę przyzwyczajenia dotyczące korzystania z komputera. Ponieważ pochodzę z kręgów informatycznych miałem okazję poznać wiele osób, które korzystają z komputera na różne sposoby, w tym różna jest ich dynamika poruszania kursorem. Krótko mówiąc, niektórzy mają szersze, szybkie ruchy myszą, a niektórzy korzystają z niej bardziej powolnymi, dokładnymi poruszeniami. Myślę, że osoby różniące się pod tym względem wypadłyby w opisanym teście zasadniczo różnie.

Dosyć pochopnym wydaje mi się też założenie, że odległość „ludzika” od środka zdjęcia wyrażona w pikselach może być miarą dystansu psychologicznego. Nawet, jeśli przeprowadzilibyśmy badanie każąc ludziom podchodzić bliżej lub dalej do zdjęć (pozycjonować własne ciało względem zdjęć), to różnice indywidualne w promieniu sfery prywatnej zaburzyłyby wyniki dotyczące odczuwanego dystansu psychologicznego. Przenosząc badanie na ekran dodatkowo tylko zaburzamy wynik.

Na koniec wreszcie -- nie wiemy nic o zastosowanych zdjęciach poszczególnych grup etnicznych. Pokazując Europejczyków w brudnych wioskach i Afroamerykanów w garniturach na pewno uzyskalibyśmy inny wynik niż jakbyśmy pokazali Afroamerykanów w niepochlebnej materialnie sytuacji i Europejczyków w ubiorze sugerującym majętność. Twórcy badania założyli, że pokazane zdjęcia różnią się tylko etnicznością modela, co z punktu widzenia badanego wcale nie musiało tak wyglądać.

\end{document}